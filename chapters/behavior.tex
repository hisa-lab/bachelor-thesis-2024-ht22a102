%! TEX root = ../main.tex
\documentclass[main]{subfiles}

\begin{document}

\chapter{アプリケーションの動作}
\label{cha:behavior}

本章では開発したウェブアプリケーションに実装した機能について実際に使用した際の動作について説明する。
具体的には「メインページ」「目安箱」「スレッド掲示板」の順に画面遷移と入力例を用いながら説明を行う。


\section{メインページ}

前章\ref{cha:behavior}章で述べたように、ユーザーがまず初めに目にする画面である。
ここでは画像〇〇に示した通り、「目安箱」「スレッド掲示板」のみが表示されている。
どちらかをクリックすることで、目的の機能へページが遷移する。
ここでは「目安箱」をクリックしたとする。

%メインページの画像
%目安箱のページ

\section{目安箱とスレッド掲示板}

「目安箱」に遷移した画面を示す。 %目安箱のページ
ここでは既に投稿された意見や提案を、簡略化し表示する。
また、表示されている意見の横には「賛同者ボタン」の現在のカウント数を表示する。

表示されている意見をクリックすることで、投稿された内容の全文を確認することができる。 %全文の一例の画像
このページで既存の意見の内容を確認し、ユーザーが賛同する場合はページ下部にある「賛同者ボタン」を押す。
こうすることで、「賛同者」をカウントし数を表示。
また、学校側から返信が行われたものは「目安箱のページ」で示した通りの表示がされる。
学校側は意見全文が表示される「全文の一例の画像」で「返信ボタン」をクリックすることで、記入フォームが出現し返信が可能になる。

また、「目安箱ページ」と「全文の一例の画像」の右上部にある「新規投稿ボタン」を押すことで自身の意見や提案を新たに送信することが可能。
「新規投稿ボタン」をクリックすることで、記述フォームと送信ボタンが出現。
記述フォームへ自身の意見や提案を記入後、送信ボタンをクリックすることで投稿が可能になる。
送信した際、
%記入した画像
%送信した画像

スレッド掲示板を選択した画面を示す。 %スレッド掲示板のページ
こちらも「目安箱」同様に既に投稿されたスレッドを一覧にし、表示。
スレッド掲示板も「目安箱」同様に、気になったスレッドタイトルをクリックすることでスレッド内容を確認することが出来る。
%既存のスレッドに参加した画像
「既存のスレッドに参加した画像」では会話内容を確認し、図下部のフォームに記入を行い、送信ボタンをクリックすることで会話へ参加できる。
%フォームへ記入した画像
%記述を送信した画像

また、「目安箱」同様に「スレッド掲示板のページ」「既存のスレッドに参加した画像」の右上部に「新規スレッド作成ボタン」がある。
これをクリックすることで、スレッドタイトルを記入し新たにスレッドが作成できる。

スレッド掲示板を利用し、学校側へ意見や提案を行うことになった場合、スレッド下部にある「学校へ提案ボタン」を押すことで
「目安箱」の新規投稿画面へ遷移。
%学校へ提案ボタン
%目安箱の送信フォーム

「目安箱」ページで新たに投稿を行う場合と異なるのは、スレッド掲示板から「目安箱」へ投稿を行う際は
記入フォーム初めに「【スレッド作成済み】」と自動入力されている点である。
この表記があるものは、意見全文の下部に当スレッドへ移るジャンプリンクが表示される。
そのため「目安箱」内に自身と同一の意見があり、【スレッド作成済み】のものであれば、利用者その結論に至るまでの過程を確認することができる。
これにより、まとまった意見の過程を見て、新たに意見を投稿するかを決めれる。



実際にアプリケーションを使用した流れを図示しながら説明を行う
・メインページ(目安箱ページへの遷移、スレッド掲示板ページへの遷移)
・目安箱内での操作(送信された意見の確認、賛同者ボタンの動作、新規投稿の動作、返信の動作)
・スレッド掲示板(既存スレッドへの参加、新規スレッドの作成、目安箱への送信)

\end{document}