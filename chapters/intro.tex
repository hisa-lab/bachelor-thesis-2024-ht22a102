
%! TEX root = ../main.tex
\documentclass[main]{subfiles}

\begin{document}

\chapter{はじめに}
\label{cha:intro}

2021年9月にキャンパス間の京阪バス運行が廃止になった。
京阪バスを利用していた生徒の中には、キャンパスではなく停留所から利用している者もいた % ????
キャンパス間の移動については学校側から代替案を出されたが、元々京阪バスが通っていた停留所から利用していた者に対する案は出されていない。
こうした学生生活における悩みが出た際、学生は友人へ相談することが一般的な流れである。
% こうしたとあるけど、上に書いてるのは、悩みではない。。 例がおかしい。
しかし、必ずしも友人がその悩みに対する解決策を持っているとは限らない。
ただ、似たような悩みを持ちやすい同じ大学のコミュニティ内で相談することが出来れば、解決策を見つけやすいと考えた。
% 考えたはやめよう。。。
% https://scrapbox.io/hisalab/%E8%AB%96%E6%96%87%E3%81%A7%E3%81%AF%E3%81%8B%E3%81%8B%E3%81%AA%E3%81%84%E8%A8%80%E8%91%89

また、京阪バスの運行廃止が生徒へ伝達されたのは夏季休暇中の2024年7月である。
友人や学内での相談の結果、学校側へ意見や提案を伝えるには「目安箱」を利用する必要がある。
現在、大阪電気通信大学において学校側へ意見を伝える「目安箱」は紙で投函する方法である。
そのため、意見や提案を伝えるためにはわざわざ学校へ行かなければならない。
さらに、紙で投函する方法では意見がどのように対応されているか生徒にはわからない。
これから対応するものなのか、現在対応している最中のものなのか、そもそも学校側としては対応できないものなのか、といったことすらも生徒にはわからない。
これが分からなければ対応が完了し生徒に影響があるまで、生徒は学校の対応に不安を感じる。
% 学生といったり、生徒といったり、表記にぶれがあります。みなおしてからだしましょう。
% 君のアプリケーションだとて、学校側の対応がわかるんですか??それをどう解決するのか説明がないです。

このような問題を解決したいと考え「意見共有と相談支援を目的としたウェブアプリケーション」の開発に取り組む。
上記の問題を解決するため、学校側へ意見を伝える「目安箱」の機能と、学内の生徒へ相談する「スレッド方式掲示板」の機能を実装する。
「目安箱」と「スレッド掲示板」の機能を組み合わせることで、学生同士で相談し解決策が無い場合は、意見をまとめすぐに学校側へ意見や提案を送信することが可能になる。
また、「目安箱」で送信された意見等は「賛同ボタン」があり、一意見に対してどの程度賛同者がいるかを確認できる。
これにより、学校側は各意見が生徒にとってどの程度重要か確認できる。

既存の掲示板アプリケーションにプラグインや拡張機能を用いて意見をまとめる機能を実装することは可能である。
しかし、導入の手間があり気軽に使いづらい。 % 導入するのはユーザではない。おかしい。
また、個人で連絡をとるチャット機能など不要に感じる機能があった。
チャット機能が不要に感じる理由としては、閉じた空間のチャット内で解決策が出された場合、今後同じような悩みを持つ者が現れた際に解決策を確認できないためである。

本論文の構成は以下の通りである。まず \ref{cha:related} 章では、関連研究について述べる。さらに、
\ref{cha:exfig} 章では、図表の作成の仕方を述べる。
最後に、\ref{cha:conclusion} 章では、本論文のまとめと今後の課題について述べる。



\end{document}