
%! TEX root = ../main.tex
\documentclass[main]{subfiles}

\begin{document}

\chapter{はじめに}
\label{cha:intro}

2024年9月にキャンパス間の京阪バス運行が廃止になった。
キャンパス間の移動については学校側から「各キャンパスにのみ停まるシャトルバスの運行」という代替案を出されたが、元々京阪バスが通っていた停留所から利用していた者に対する案は出されていない。
そのため、停留所から利用していた学生は新たに通学ルートを模索しなくてはならくなった。
こうした際、同じ状況になった友人へ相談するケースが多い。
しかし、必ずしも友人が同じような状況になるわけではない。
このような場合、同じ大学のコミュニティ内で相談することが出来れば、解決策を見つけやすい。

また上記の状況で友人や学内への相談の結果、「京阪バスが停まっていた停留所にもシャトルバスを停めてほしい」といった意見や、
その他の提案は、「目安箱」を通して学校側へ伝えることが出来る。
しかし、「目安箱」を通して意見や提案を伝えた際、どのように対応されているか学生にはわからない。
これから対応するものなのか、現在対応している最中のものなのか、そもそも学校側としては対応できないものなのか、といったことすらも学生にはわからない。

このような問題を解決したいと考え「意見共有と相談支援を目的としたウェブアプリケーション」の開発に取り組む。
上記の問題を解決するため、学校側へ意見を伝える「目安箱」の機能と、学内の学生へ相談する「スレッド方式掲示板」の機能を実装する。
「目安箱」と「スレッド掲示板」の機能を組み合わせることで、学生同士で相談し解決策が無い場合は、意見をまとめすぐに学校側へ意見や提案を送信することが可能になる。
%% ここはよくなっている。これだったらあってもいいと思う内容。(これはただのコメント)
また、「目安箱」で送信された意見等は「賛同ボタン」があり、一意見に対してどの程度賛同者がいるかを確認できる。
これにより、学校側は各意見が学生にとってどの程度重要か確認できる。
また、送信された意見等へ対応の是非や中途報告を学生へ伝えるため、当アプリケーションでは返信することが可能。
これには、学生の学校に対する不安感を拭う効果がある。


本論文の構成は以下の通りである。まず \ref{cha:related} 章では、関連研究について述べる。
\ref{cha:overview}章では、アプリケーションの概要について述べ、
\ref{cha:behavior}章でアプリケーションの動作について述べる。
そして、\ref{cha:evaluation}章でアプリケーションの評価について述べる
最後に、\ref{cha:conclusion} 章では、本論文のまとめと今後の課題について述べる。




\end{document}