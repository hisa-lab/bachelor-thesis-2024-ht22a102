
%! TEX root = ../main.tex
\documentclass[main]{subfiles}

\begin{document}

\chapter{はじめに}
\label{cha:intro}

2024年9月にキャンパス間の京阪バス運行が廃止になった。
キャンパス間の移動については学校側から「各キャンパスにのみ停まるシャトルバスの運行」という代替案を出されたが、元々京阪バスが通っていた停留所から利用していた者に対する案は出されていない。
そのため、停留所から利用していた学生は新たに通学ルートを模索しなくてはならくなった。
こうした際、同じ状況になった友人へ相談するケースが多い。
% 流れとか、口語的すぎる。。。。。。
しかし、必ずしも友人が同じような状況になるわけではない。
このような場合、同じ大学のコミュニティ内で相談することが出来れば、解決策を見つけやすい。

また上記の状況で友人や学内への相談の結果、学校側へ「京阪バスが停まっていた停留所にもシャトルバスを停めてほしい」といった意見や、その他の提案を伝えるには「目安箱」を利用する必要がある。
% なぜ、学校側に意見を伝える話になっているの???友人に通学路の相談をするという話から、飛躍しすぎでは?
現在、大阪電気通信大学において学校側へ意見を伝える「目安箱」は紙で投函する方法である。
京阪バスの運行廃止が学生へ伝達された2024年7月18日(木)は、前期授業が終了しており定期試験期間の最中である。
定期試験期間内であれば、試験を受けるついでに「目安箱」に意見を投函することはできる。
しかし、定期試験期間が終われば学生は意見や提案を学校側へ伝えたい際は、わざわざ学校へ行かなければならない。
% 7/16-2/22 定期試験期間やで。。。定期券購入してないのは、別の問題やろう。
% 説明がくるしすぎる。。。
また現在、意見がどのように対応されているか学生にはわからない。
% 紙でしてるからではない。単にどのように対応してるか知らせるようになってないだけ。問題点は、紙ではない。
これから対応するものなのか、現在対応している最中のものなのか、そもそも学校側としては対応できないものなのか、といったことすらも学生にはわからない。
これが分からなければ対応が完了し学生に影響があるまで、学生は学校の対応に不安を感じる。

このような問題を解決したいと考え「意見共有と相談支援を目的としたウェブアプリケーション」の開発に取り組む。
上記の問題を解決するため、学校側へ意見を伝える「目安箱」の機能と、学内の学生へ相談する「スレッド方式掲示板」の機能を実装する。
「目安箱」と「スレッド掲示板」の機能を組み合わせることで、学生同士で相談し解決策が無い場合は、意見をまとめすぐに学校側へ意見や提案を送信することが可能になる。
%% ここはよくなっている。これだったらあってもいいと思う内容。(これはただのコメント)
また、「目安箱」で送信された意見等は「賛同ボタン」があり、一意見に対してどの程度賛同者がいるかを確認できる。
これにより、学校側は各意見が学生にとってどの程度重要か確認できる。
また、送信された意見等へ対応の是非や中途報告を学生へ伝えるため、当アプリケーションでは返信することが可能。
これには、学生の学校に対する不安感を拭う効果がある。

既存の掲示板アプリケーションにプラグインや拡張機能を用いて意見をまとめる機能を実装することは可能である。
そのため既存の掲示板アプリケーションを調整すれば、当目的は達成できる。
しかし既存の掲示板アプリケーションには、個人で連絡をとるチャット機能など不要に感じる機能があった。% チャットはどこからでてきた???
チャット機能が不要に感じる理由としては、閉じた空間のチャット内で解決策が出された場合、今後同じような悩みを持つ者が現れた際に解決策を確認できないためである。
当アプリケーションでは、これら不要な機能を排除し「意見共有と相談支援」のみに重点を置いた。

本論文の構成は以下の通りである。まず \ref{cha:related} 章では、関連研究について述べる。
さらに、\ref{cha:exfig} 章では、図表の作成の仕方を述べる。
最後に、\ref{cha:conclusion} 章では、本論文のまとめと今後の課題について述べる。




\end{document}