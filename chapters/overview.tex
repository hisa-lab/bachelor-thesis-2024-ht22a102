%! TEX root = ../main.tex
\documentclass[main]{subfiles}

\begin{document}

\chapter{開発したウェブアプリケーションの構成}
\label{cha:overview}
\section{アプリケーションの構成}

開発したウェブアプリケーションのシステム構成図を示す %システム構成図を挿入(PPで作成したもの)
当アプリケーションではフロントエンドにHTML、Booststrap、React、サーバーサイドにNode.jsを使用して構成される。
フロントエンドでは、ユーザーが使用する「目安箱」「スレッド掲示板」の機能を提供。
Reactのコンポーネントベースの設計を用いたため、ユーザーインターフェースをモジュール化し、拡張性を向上させた。
反対にサーバーサイドでは、Expressサーバーを用いて「目安箱」に送信されたデータ、スレッド掲示板のデータの保管を行った。

\section{アプリケーションの機能}
当研究で開発したアプリケーションに実装された機能について述べる。 %アプリ画面の全体を挿入
当アプリケーションは主に、「メインページ」「目安箱」「スレッド掲示板」の機能によって構成される。
各機能を以下で説明する。

\section{メインページ}

当アプリケーションの「メインページ」はユーザーが一番初めに目にする画面である。
当アプリケーションを利用するユーザーが「目安箱」「スレッド掲示板」のどちらを利用するかを決めてもらう。
「メインページ」の機能としては上記の通りのため、当アプリケーションを開いて以降は目にすることはない。

\section{目安箱}

当アプリケーションの「目安箱」は、学生の意見や提案の確認、新規投稿が可能になっている。
既に送信されている意見を確認し、自身が送信したい内容とかぶっていなければ、新規投稿を行う。

当機能には〇〇を使用している。
送信した日時を〇〇で取得し、バックエンドのExpressサーバーでSQLiteを持ち多データベースへ保管される。

\section{スレッド掲示板}

当アプリケーションの「スレッド掲示板」は、スレッド方式の掲示板として悩んでいることなどについて相談することが出来る。


アプリケーションの機能を図示しながら説明を行う
・メインページ
・目安箱
・スレッド掲示板

\end{document}